@article{Aditya2023,
   abstract = {Li-ion cell is a popular electrochemistry for energy storage devices in electric vehicles and portable electronics. This is mainly due to their relatively high energy-density and power-density compared to other types of battery chemistry. In a battery pack, several li-ion cells are interconnected in a series-parallel fashion to meet the current and voltage demands of the load. Interconnecting wires or wire-bonds add parasitic resistance to the current path, contributing to power loss. Therefore, it is essential to include wire-bond losses in the battery model to get an accurate prediction of battery performance. In this paper, the authors have developed a 2RC model of li-ion cells, taking into account the wire-bond losses. Simulation was performed in MATLAB Simulink and the simulated results were verified with hardware results obtained using a li-ion battery cycler.},
   author = {Kunwar Aditya},
   doi = {10.1109/ICCC57789.2023.10165007},
   isbn = {9798350334128},
   journal = {2023 International Conference on Control, Communication and Computing, ICCC 2023},
   keywords = {2RC Model,Electric Vehicles,Electrical Equivalent Circuit Model,Energy Storage,Li-Ion Modelling},
   publisher = {Institute of Electrical and Electronics Engineers Inc.},
   title = {Simulation and Validation of a Precise 2RC Model of Lithium-Ion cell Incorporating Wire-bond Losses},
   url = {https://www.researchgate.net/publication/372109735_Simulation_and_Validation_of_a_Precise_2RC_Model_of_Lithium-Ion_cell_Incorporating_Wire-bond_Losses},
   year = {2023}
}
@misc{Valizadeh2024,
   abstract = {Machine Learning has garnered significant attention in lithium-ion battery research for its potential to revolutionize various aspects of the field. This paper explores the practical applications, challenges, and emerging trends of employing Machine Learning in lithium-ion battery research. Delves into specific Machine Learning techniques and their relevance, offering insights into their transformative potential. The applications of Machine Learning in lithium-ion-battery design, manufacturing, service, and end-of-life are discussed. The challenges including data availability, data preprocessing and cleaning challenges, limited sample size, computational complexity, model generalization, black-box nature of Machine Learning models, scalability of the algorithms for large datasets, data bias, and interdisciplinary nature and their mitigations are also discussed. Accordingly, by discussing the future trends, it provides valuable insights for researchers in this field. For example, a future trend is to address the challenge of small datasets by techniques such as Transfer Learning and N-shot Learning. This paper not only contributes to our understanding of Machine Learning applications but also empowers professionals in this field to harness its capabilities effectively.},
   author = {Alireza Valizadeh and Mohammad Hossein Amirhosseini},
   doi = {10.1007/s42979-024-03046-2},
   issn = {26618907},
   issue = {6},
   journal = {SN Computer Science},
   keywords = {Artificial intelligence,Data-driven approach,Lithium-ion battery,Machine learning},
   month = {8},
   publisher = {Springer},
   title = {Machine Learning in Lithium-Ion Battery: Applications, Challenges, and Future Trends},
   volume = {5},
   year = {2024}
}
@article{Piruzjam2024,
   abstract = {The Doyle–Fuller–Newman (DFN) model serves as a fundamental electrochemical modeling framework widely used in the field of lithium-ion batteries, especially in its pseudo-two-dimensional (P2D) formulation. Despite its widespread use, the DFN model imposes significant computational demands, especially for tasks such as cell characterization and state estimation in electric vehicles’ battery management systems (BMS). To mitigate these computational challenges, researchers have developed several reduced-order models. In this study, we tackle the associated computational efforts starting from the governing DFN equations. The analytical solutions for the transport equations in the electrode and the electrolyte are derived, considering generic time-varying boundary conditions under certain assumptions. The obtained exact solutions are then used to develop an analytical model for a single particle model with electrolyte dynamics (SPME) which provides higher predictive accuracy compared to a single particle model (SPM), especially in high C-rate scenarios, which are critical in many applications, including electric vehicles. Furthermore, the exact solution of lithium diffusion in active material particles is incorporated into the standard P2D model. This integration leads to a semi-analytical variant of the P2D model (SAP2D). A remarkable advantage of these analytical solutions is the significantly lower computational cost compared to corresponding numerical approaches. Spatial discretization becomes unnecessary, and the solutions can be obtained by a few explicit function evaluations. Moreover, when the time-varying boundary conditions are known, the need for time stepping is also eliminated.},
   author = {Javid Piruzjam and Guangming Liu and Lukas Rubacek and Marcus Frey and Thomas Carraro},
   doi = {10.1016/j.electacta.2024.144259},
   issn = {00134686},
   journal = {Electrochimica Acta},
   keywords = {Analytical SPME,Electric vehicles,Generic time-varying boundary condition,Lithium-ion batteries,Semi-analytical P2D},
   month = {7},
   publisher = {Elsevier Ltd},
   title = {On the analytical solution of single particle models and semi-analytical solution of P2D model for lithium-ion batteries},
   volume = {492},
   year = {2024}
}
@article{Kawahara2023,
   abstract = {Battery performance prediction techniques based on machine learning (ML) models and lithium-ion battery (LIB) data collected in the real world have received much attention recently. However, poor extrapolation accuracy is a major challenge for ML models using real-world data, as the data frequency distribution can be uneven. Here, we have investigated the extrapolation accuracy of the ML models by using artificial data generated with an electrochemical simulation model. Specifically, we set a lower open circuit voltage (OCV) limit for the training data and generated data limited to the higher state of charge (SOC) region to train the voltage prediction model. We have validated the root mean squared error (RMSE) of the voltage for the test data at several lower OCV limit settings and defined the average+3 standard deviations of them as an evaluation metric. Eight representative ML models were evaluated, and it was found that the multilayer perceptron (MLP) showed an accuracy of 92.7 mV, which was the best extrapolation accuracy. We also evaluated models with published experimental data and found that the MLP had an accuracy of 102.4 mV, reconfirming that it had the best extrapolation accuracy. We also found that MLP was robust to changes in the data of interest since the accuracy degradation when changing from simulation to experimental data was as small as a factor of 1.1. This result shows that MLP can achieve higher voltage prediction accuracy even when collecting data for comprehensive SOC conditions is difficult.},
   author = {Takuma Kawahara and Koji Sato and Yuki Sato},
   doi = {10.1155/2023/5513446},
   issn = {1099114X},
   journal = {International Journal of Energy Research},
   publisher = {Wiley-Hindawi},
   title = {Battery Voltage Prediction Technology Using Machine Learning Model with High Extrapolation Accuracy},
   volume = {2023},
   year = {2023}
}
@article{Li2018,
   abstract = {State of Health (SOH) estimation of lithium ion batteries is critical for Battery Management Systems (BMSs) in Electric Vehicles (EVs). Many estimation techniques utilize a battery model; however, the model must have high accuracy and high computational efficiency. Conventional electrochemical full-order models can accurately capture battery states, but they are too complex and computationally expensive to be used in a BMS. A Single Particle (SP) model is a good alternative that addresses this issue; however, existing SP models do not consider degradation physics. In this work, an SP-based degradation model is developed by including Solid Electrolyte Interface (SEI) layer formation, coupled with crack propagation due to the stress generated by the volume expansion of the particles in the active materials. A model of lithium ion loss from SEI layer formation is integrated with an advanced SP model that includes electrolytic physics. This low-order model quickly predicts capacity fade and voltage profile changes as a function of cycle number and temperature with high accuracy, allowing for the use of online estimation techniques. Lithium ion loss due to SEI layer formation, increase in battery resistance, and changes in the electrodes’ open circuit potential operating windows are examined to account for capacity fade and power loss. In addition to the low-order implementation to facilitate on-line estimation, the model proposed in this paper provides quantitative information regarding SEI layer formation and crack propagation, as well as the resulting battery capacity fade and power dissipation, which are essential for SOH estimation in a BMS.},
   author = {J. Li and K. Adewuyi and N. Lotfi and R. G. Landers and J. Park},
   doi = {10.1016/j.apenergy.2018.01.011},
   issn = {03062619},
   journal = {Applied Energy},
   keywords = {Battery management systems,Capacity degradation,On-line estimation,Power loss,Single particle model,State of health},
   month = {2},
   pages = {1178-1190},
   publisher = {Elsevier Ltd},
   title = {A single particle model with chemical/mechanical degradation physics for lithium ion battery State of Health (SOH) estimation},
   volume = {212},
   year = {2018}
}
@article{Xue2023,
   abstract = {As power sources for electric vehicles, lithium-ion batteries (LIBs) have many advantages, such as high energy density and wide temperature range. In the algorithm design process for LIBs, various battery models with different model structures are needed, among which the electrochemical model is widely used due to its high accuracy. However, the electrochemical model is composed of multiple nonlinear partial differential equations (PDEs) that make the simulating process time-consuming. In this paper, a physics-informed neural network single-particle model (PINN SPM) is proposed to improve the accuracy of the single-particle model (SPM) under high C-rates, while ensuring high solving speed. In PINN SPM, an SPM-Net is designed to solve the distribution of lithium-ion concentration in the electrolyte. In the neural network learning process, a loss function is designed based on the physical constraints brought by the PDEs, which reduces the error of the neural network under dynamic working conditions. Finally, the PINN SPM proposed in this paper can achieve a maximum relative error of up to 1.2% compared with the high-fidelity data generated from the P2D model under various conditions. Additionally, the PINN SPM is 20.8% faster than traditional numerical solution methods with the same computational resources.},
   author = {Chenyu Xue and Bo Jiang and Jiangong Zhu and Xuezhe Wei and Haifeng Dai},
   doi = {10.3390/batteries9100511},
   issn = {23130105},
   issue = {10},
   journal = {Batteries},
   keywords = {electrochemical model,electrolyte dynamics,lithium-ion battery,physic-informed neural network,single particle model},
   month = {10},
   publisher = {Multidisciplinary Digital Publishing Institute (MDPI)},
   title = {An Enhanced Single-Particle Model Using a Physics-Informed Neural Network Considering Electrolyte Dynamics for Lithium-Ion Batteries},
   volume = {9},
   year = {2023}
}
@article{Xia2024,
   abstract = {Monitoring the state of the battery, including the state of charge (SOC) and state of health (SOH), is crucial for ensuring the safety and reliability of electrical equipment. The paper presents a novel hybrid network that combines nonlinear autoregressive model with exogenous inputs (NARX) and DS-attention. The proposed DS-attention method establishes a robust mapping relationship between inputs and outputs, it is a specialized method of the recurrent neural network that enhances the estimation performance by incorporating division function and self-adaptive function into the attention mechanism. The division function is designed to efficiently differentiate between exogenous inputs and state outputs, thereby minimizing the potential for cross-interference between them, and the self-adaptive function optimizes the query features within the attention mechanism. The results demonstrate that the hybrid method of NARX and DS-attention achieves higher prediction accuracy for SOH and SOC estimation of lithium-ion batteries. Specifically, compared with the best-performing similar methods, the proposed hybrid method enhanced the prediction accuracy by 22.9%, 60.7%, and 51.2% across three different SOH datasets, respectively. In terms of long sequence SOC forecasting, the improvements in prediction accuracy under two different working conditions are 82.5% and 60.5%, respectively. The proposed algorithm optimizes the attention mechanism based on the characteristics of the NARX, resulting in higher estimation accuracy for predicting the state of lithium batteries.},
   author = {Zhehao Xia and Yizhong Wu},
   doi = {10.1016/j.ijoes.2024.100632},
   issn = {14523981},
   issue = {7},
   journal = {International Journal of Electrochemical Science},
   keywords = {DS-attention,Nonlinear auto-regressive models with exogenous inputs neural network,Recurrent neural network,State of health and state of charge of lithium-ion batteries},
   month = {7},
   publisher = {Elsevier B.V.},
   title = {A hybrid network of NARX and DS-attention applied for the state estimation of lithium-ion batteries},
   volume = {19},
   year = {2024}
}
@article{Chen2022,
   abstract = {The battery analytical model often plays an important role in accurately estimating the online state-of-charge of the battery. The improved gas-liquid dynamics battery model is proposed to simulate the physicochemical behaviors of a lithium-ion battery. The estimation equations of both iterative open-circuit voltage and terminal voltage are deduced according to this model. A strong robust state-of-charge estimation method is designed without coupling optimization algorithm based on the iterative open-circuit voltage estimation equation. Experimental results of the Li(NiMnCo)O2 batteries under the Dynamic Stress Test cycle, the New European Driving Cycle, the Federal Urban Driving Schedule cycle, the Urban Dynamometer Driving Schedule cycle and the constant current test confirm the efficacy of the proposed approach. Moreover, this method has excellent robustness against the initial error of 50% state-of-charge which is eliminated within 6 s under five working conditions and it provides a reliable state-of-charge estimation for the sampling data of different sampling periods between a second and 60 s.},
   author = {Biao Chen and Haobin Jiang and Xijia Chen and Huanhuan Li},
   doi = {10.1016/j.energy.2021.122008},
   issn = {03605442},
   journal = {Energy},
   keywords = {Gas-liquid dynamics battery model,Lithium-ion battery,Online estimation,State-of-charge estimation},
   month = {1},
   publisher = {Elsevier Ltd},
   title = {Robust state-of-charge estimation for lithium-ion batteries based on an improved gas-liquid dynamics model},
   volume = {238},
   year = {2022}
}
@article{Wang2021,
   abstract = {To accurately estimate the state of charge (SOC) of the aged batteries, the capacity estimation must not be ignored. Based on the battery charging data, this paper proposes a co-estimation model to estimate the SOC and capacity for lithium-ion batteries. First, a new health indicator of capacity is extracted based on the charging data of lithium batteries; second, the capacity is estimated by least squares support vector machine (LSSVM). The results are recorded based on a memory gate and used as the input of SOC estimation. Third, a moving window method is adopted to address the long-term dependency loss problem of the recurrent neural network (RNN), and a co-estimation model is obtained. Finally, the proposed model is compared with other models. The results show that the proposed model performs better. The maximum RMSE of the proposed model is 0.85%, and the computational cost of the proposed model is below 5 s.},
   author = {Qiao Wang and Min Ye and Meng Wei and Gaoqi Lian and Chenguang Wu},
   doi = {10.1016/j.egyr.2021.10.095},
   issn = {23524847},
   journal = {Energy Reports},
   keywords = {Capacity estimation,Lithium-ion batteries,Moving window,Recurrent neural network,State of charge,Support vector machine},
   month = {11},
   pages = {7323-7332},
   publisher = {Elsevier Ltd},
   title = {Co-estimation of state of charge and capacity for lithium-ion battery based on recurrent neural network and support vector machine},
   volume = {7},
   year = {2021}
}
@article{Ali2024,
   abstract = {Physics-based electrochemical battery models, such as the Doyle-Fuller-Newman (DFN) model, are valuable tools for simulating Li-ion battery behavior and understanding internal battery processes. However, the complexity and computational demands of such models limit their applicability for battery management systems and long-term aging simulations. Reduced-order models (ROMs), such as the Extended Single Particle Model (ESPM), Single Particle Model (SPM) and Polynomial and Padé approximations, here all referred to as simplifications, lead to faster computational speeds. Choosing the appropriate simplification method for a specific cell type and operating condition is a challenge. This study investigates the simulation accuracy and calculation speed of various simplifications for high-energy (HE) and high-power (HP) batteries at different current loading conditions and compares those to the full-order DFN model. The results indicate that among the ROMs, the ESPM consistently offers the best combination of high computational speed and relatively good accuracy in most conditions in comparison to the full-order DFN model. Among the approximations, higher-order polynomial approximation, third and fourth-order Padé approximation perform the best in terms of accuracy. The higher-order polynomial approximation shows an advantage in terms of computing speed, while the fourth-order Padé approximation achieves the highest overall accuracy among the different approximations.},
   author = {Haider Adel Ali Ali and Luc H.J. Raijmakers and Kudakwashe Chayambuka and Dmitri L. Danilov and Peter H.L. Notten and Rüdiger A. Eichel},
   doi = {10.1016/j.electacta.2024.144360},
   issn = {00134686},
   journal = {Electrochimica Acta},
   keywords = {Battery modeling,Extended single particle model (ESPM),Pseudo-two dimensional (P2D),Reduced-order models,Single particle model (SPM)},
   month = {7},
   publisher = {Elsevier Ltd},
   title = {A comparison between physics-based Li-ion battery models},
   volume = {493},
   year = {2024}
}
@article{Drummond2019,
   abstract = {An observer is designed for the Doyle-Fuller-Newman electrochemical model of a Li-ion battery under the assumption of constant ion exchange current density and MacInnes’ equation being concentration independent. For this nonlinear system, the observer is shown to experience both exponential convergence and a degree of robustness. This result extends the class of electrochemical model for battery observer design, generating improved estimates for the state-of-charge and for the spatial variation of the overpotential across an electrode.},
   author = {R. Drummond and S. R. Duncan},
   doi = {10.1016/j.est.2019.02.012},
   issn = {2352152X},
   journal = {Journal of Energy Storage},
   keywords = {Electrochemical models,Li-ion batteries,Observer design,State-of-charge estimation},
   month = {6},
   pages = {250-257},
   publisher = {Elsevier Ltd},
   title = {Observer design for the Doyle–Fuller–Newman Li-ion battery model without electrolyte dynamics},
   volume = {23},
   year = {2019}
}
@article{Chen2021,
   abstract = {The capacity loss and cycling aging of lithium-ion batteries at high (dis)charging rate (C-rate) hinders the development of emerging technologies. To improve the performance of Li-ion batteries, it is important to understand the coupling effect of the mechanical behaviors and the electrochemical response of electrodes, as the capacity loss and cycling aging are related to the mechanics of electrodes during (dis)charging. Many studies have formulated the distribution of stress, strain and lithium-ion fraction of electrodes during lithiation/delithiation. However, few of them reported a self-consistent formulation that contains mechanical-diffusional-electrochemical coupling effects, solid viscosity, and diffusion-induced creep for an electrode with large deformation under non-equilibrium process. This paper considers the electrode of a Li-ion battery as a solid solution system. Based on continuum mechanics, non-equilibrium thermodynamics and variational theory, we develop a generalized theory to describe the variations of stress distribution, electrode material deformation and lithium-ion fractions of the solid solution system over a non-equilibrium process. The finite deformation, mass transfer, phase transformation, chemical reaction and electrical potential of the system are coupled with each other in a fully self-consistent formulation. We apply the developed theory to numerically simulate a Sn anode particle using the finite difference method. Our results compare the influences of different C-rates on the non-equilibrium process of the anode particle. Higher C-rate corresponds to stronger dissipation effects including faster plastic deformation, larger viscous stress, more polarization in the electrical potential, longer relaxation time and less electrical energy. With the formulation and simulation of the non-equilibrium process, this study refines our understanding of the mechanical-diffusional-electrochemical coupling effect in Li-ion batteries with high C-rate.},
   author = {Hongjiang Chen and Hsiao Ying Shadow Huang},
   doi = {10.1016/j.ijsolstr.2020.11.014},
   issn = {00207683},
   journal = {International Journal of Solids and Structures},
   keywords = {Continuum mechanics,Finite deformation,Li-ion batteries,Modeling and simulation,Non-equilibrium thermodynamics,Solid solution system},
   month = {3},
   pages = {124-142},
   publisher = {Elsevier Ltd},
   title = {Modeling and simulation of the non-equilibrium process for a continuous solid solution system in lithium-ion batteries},
   volume = {212},
   year = {2021}
}
@inproceedings{Li2019,
   abstract = {Battery is the bottleneck technology of electric vehicles. The complex chemical reactions inside the battery are difficult to monitor directly. The establishment of a precise mathematical model for the battery is of great significance in ensuring the secure and stable operation of the battery management system. First of all, a data cleaning method based on machine learning is put forward, which is applicable to the characteristics of big data from batteries in electric vehicles. Secondly, this paper establishes a lithium-ion battery model based on deep learning algorithm and the error of model based on different algorithms is compared. The data of electric buses are used for validating the effectiveness of the model. The result shows that the data cleaning method achieves good results, in the case of the terminal voltage missing, the mean absolute percentage error of filling is within 4%, and the battery modeling method in this paper is able to simulate the battery characteristics accurately, and the mean absolute percentage error of the terminal voltage estimation is within 2.5%.},
   author = {Shuangqi Li and Jianwei Li and Hongwen He and Hanxiao Wang},
   doi = {10.1016/j.egypro.2018.12.046},
   issn = {18766102},
   booktitle = {Energy Procedia},
   keywords = {battery management,bigdata,deeplearning,electric vehicle,lithium-ion power battery,modeling},
   pages = {168-173},
   publisher = {Elsevier Ltd},
   title = {Lithium-ion battery modeling based on Big Data},
   volume = {159},
   year = {2019}
}
@article{Khalfi2021,
   abstract = {The Box–Jenkins model is a polynomial model that uses transfer functions to express re-lationships between input, output, and noise for a given system. In this article, we present a Box– Jenkins linear model for a lithium-ion battery cell for use in electric vehicles. The model parameter identifications are based on automotive drive-cycle measurements. The proposed model prediction performance is evaluated using the goodness-of-fit criteria and the mean squared error between the Box–Jenkins model and the measured battery cell output. A simulation confirmed that the proposed Box–Jenkins model could adequately capture the battery cell dynamics for different automotive drive cycles and reasonably predict the actual battery cell output. The goodness-of-fit value shows that the Box–Jenkins model matches the battery cell data by 86.85% in the identification phase, and 90.83% in the validation phase for the LA-92 driving cycle. This work demonstrates the potential of using a simple and linear model to predict the battery cell behavior based on a complex identification dataset that represents the actual use of the battery cell in an electric vehicle.},
   author = {Jaouad Khalfi and Najib Boumaaz and Abdallah Soulmani and El Mehdi Laadissi},
   doi = {10.3390/wevj12030102},
   issn = {20326653},
   issue = {3},
   journal = {World Electric Vehicle Journal},
   keywords = {Automotive drive-cycle measurements,Box–Jenkins model,Electric vehicles,Lithium-ion battery cell},
   month = {9},
   publisher = {MDPI AG},
   title = {Box–jenkins black-box modeling of a lithium-ion battery cell based on automotive drive cycle data},
   volume = {12},
   year = {2021}
}
@article{Ledovskikh2016,
   abstract = {Insertion reactions are of key importance for Li/Na-ion batteries and hydrogen storage materials. Nanosizing of these energy storage materials has been shown to have a fundamental impact on the storage properties. Predicting these properties based on rather simple thermodynamic grounds is of high importance for fundamental understanding, achieving the optimal performance of nanomaterials, as well as for the practical ability to manage battery systems. Here we report on the development of a new thermodynamic lattice gas model based on the equation of state of the energy carrier that is able to describe the impact of particle size on fundamental physical-chemical characteristics, such as the phase diagram and equilibrium potentials of energy storage materials that exhibit a first-order phase transition upon Li or H insertion. The model is based on the first-principles of chemical and statistical thermodynamics and takes into account complex structural changes taking place in energy storage materials and because of its general nature can be adapted to describe the influence of any state variable (particle size, temperature, etc.). The model is applied and validated using experimental data on different particle sizes of the LiFePO4 battery electrode material resulting in excellent agreement. The model can be used to simulate phase diagrams and predict equilibrium potential isotherms with respect to the electrode nanoparticle size. The relative simplicity of the model allows easy prediction of material properties as required by for instance advanced battery management systems.},
   author = {A. V. Ledovskikh and M. Wagemaker},
   doi = {10.1021/acs.jpcc.6b00914},
   issn = {19327455},
   issue = {20},
   journal = {Journal of Physical Chemistry C},
   month = {5},
   pages = {11192-11203},
   publisher = {American Chemical Society},
   title = {Lattice-Gas Model for Energy Storage Materials: Phase Diagram and Equilibrium Potential as a Function of Nanoparticle Size},
   volume = {120},
   year = {2016}
}
@article{Khalik2021,
   abstract = {Using electrochemistry-based battery models in battery management systems remains challenging due to their computational complexity. In this paper, we study for the first time the impact of several types of model simplifications on the trade-off between model accuracy and computation time for the Doyle–Fuller–Newman (DFN) model. As a basis for comparison, we consider, to what we refer as, the complete DFN (CDFN) model, which is a DFN model without any simplifications, and includes the concentration-dependency of parameters that have been studied in previous literature. Furthermore, we propose a highly efficient implementation of the CDFN model that leads to a considerable decrease in computation time, and is developed into a freely downloadable toolbox. This toolbox allows the user to easily toggle between the studied simplifications to make the desired trade-off between model accuracy and computation time. We compare several simplified DFN models to the single-particle-model and the CDFN model. Here, we show that with the proposed implementation, and by selectively making the proposed simplifications, as well as selectively choosing the grid parameters, a model can be obtained that has a minor impact on model accuracy, achieving a simulation time of over 5000 times faster than real-time.},
   author = {Z. Khalik and M. C.F. Donkers and H. J. Bergveld},
   doi = {10.1016/j.jpowsour.2020.229427},
   issn = {03787753},
   journal = {Journal of Power Sources},
   keywords = {Battery model implementation,Battery model simplifications,Battery modeling,Battery simulation,Battery simulation toolbox},
   month = {3},
   publisher = {Elsevier B.V.},
   title = {Model simplifications and their impact on computational complexity for an electrochemistry-based battery modeling toolbox},
   volume = {488},
   year = {2021}
}
@article{Quelin2023,
   abstract = {The design of a battery-powered system ideally requires the simultaneous sizing of all its components. To contribute to this purpose, we propose a physical approach to couple the electrical parameters of a battery equivalent-circuit-model (ECM) to the electrodes dimensions of the battery. Because it only requires non-invasive measurements, it can be easily used by system integrators. To test the proposed approach, we choose three commercial coin cells with different sizes and nominal capacities (25 mA h, 60 mA h and 120 mA h). Only the electrodes length of these cells varies, and this study is therefore focused on creating dependency-models that predict the ECM parameters values with respect to this specific dimension. The proposed ECM brings accurate voltage simulations and the dependency-models predictions are satisfactory for the three cells, with a mean accuracy of 6.3%. Thanks to the proposed approach, the parameters of any cell size and capacity can be predicted in the characterization range (between 25 mA h and 120 mA h here). It is thus a promising tool for developing custom-made cells.},
   author = {Aurélien Quelin and Nicolas Damay},
   doi = {10.1016/j.jpowsour.2023.232690},
   issn = {03787753},
   journal = {Journal of Power Sources},
   keywords = {Battery,Electrical modeling,Electrodes dimensions,Equivalent circuit model},
   month = {3},
   publisher = {Elsevier B.V.},
   title = {Coupling electrical parameters of a battery equivalent circuit model to electrodes dimensions},
   volume = {561},
   year = {2023}
}
@article{Pai2023,
   abstract = {To ensure the battery management system (BMS) operates effectively, it is quite important to accurately determine the model parameters of the equivalent circuit model (ECM) and the state of charge (SOC). Decoupled recursive least squares (DRLS) technique which separately identifies the parameters of the battery's fast and slow dynamic response impedance can improve the real-time estimation accuracy, and time-domain parameter extraction (TDPE) methods can obtain model parameters from experimental voltage data without optimizing the parameters iteratively; hence it is simple and time-effective. To have both the advantages of fast identification speed of TDPE and high accuracy of DRLS at the same time, this study proposes a parameter identification technique that combines the two methods mentioned above. Integrating the TDPE approach and DRLS technique can effectively simplify the complexity of the DRLS estimation method and improve its estimation accuracy. In addition, a direct current resistance (DCR) compensation term considering fast and slow dynamics parameters is also proposed to further improve the identification accuracy of DCR, which further enhances the SOC estimation precision. Compared with the original DRLS technique, the mean absolute percentage error (MAPE) of the modelling error obtained by the proposed method can be improved by 78.7 %/73.5 % under the federal urban driving schedule (FUDS) and the dynamic stress test (DST) test patterns for simulated results, and enhanced by 9.0 %/5.3 % under FUDS and DST test profiles for experimental results. In addition, the MAPE of the SOC error is reduced by 72.3 %/7.1 % under FUDS and DST test patterns for simulated results, and lowered by 18.8 %/8.3 % under FUDS and DST test profiles for experimental results. Those results validate the effectiveness and correctness of the proposed method.},
   author = {Hung Yu Pai and Yi Hua Liu and Song Pei Ye},
   doi = {10.1016/j.est.2023.106901},
   issn = {2352152X},
   journal = {Journal of Energy Storage},
   keywords = {Battery management system,Decoupled recursive least squares,Equivalent circuit model,State of charge,Time-domain parameter extraction},
   month = {6},
   publisher = {Elsevier Ltd},
   title = {Online estimation of lithium-ion battery equivalent circuit model parameters and state of charge using time-domain assisted decoupled recursive least squares technique},
   volume = {62},
   year = {2023}
}
@article{Mawuntu2023,
   abstract = {The transportation sector is under increasing pressure to reduce greenhouse gas emissions by decarbonizing its operations. One prominent solution that has emerged is the adoption of electric vehicles (EVs). As the electric vehicles market experiences rapid growth, the utilization of lithium-ion batteries (LiB) has become the predominant choice for energy storage. However, it is important to note that lithium-ion battery technology is sensitive to factors, like excessive voltage and temperature. Therefore, the development of an accurate battery model and a reliable state of charge (SOC) estimator is crucial to safeguard against the overcharging and over-discharging of the battery. Numerous studies have been conducted to address lithium-ion battery cell modeling and SOC estimations. These studies have explored variations in the number of RC networks within the model and different estimation methods. However, it is worth mentioning that the capacity of a single lithium-ion battery cell is relatively low and cannot be directly employed in electric vehicles. To meet the total capacity and voltage requirements for electric vehicles, multiple cells are typically connected in series or parallel configurations to form a battery pack. Surprisingly, this aspect has often been overlooked in previous research. To tackle this overlooked challenge, our study introduces a comprehensive battery pack model and an advanced Battery Management System (BMS). We then integrate these components into an electric vehicle model. Subsequently, we simulate the integrated EV-BMS model under the conditions of four different urban driving scenarios to replicate real-world driving conditions. The BMS that we have developed includes an Extended Kalman Filter (EKF)-based SOC estimation system, a mechanism for controlling coolant flow, and a passive cell-balancing algorithm. These components work together to ensure the safe and efficient operation of the battery pack within the electric vehicles.},
   author = {Nadya Novarizka Mawuntu and Bao Qi Mu and Oualid Doukhi and Deok Jin Lee},
   doi = {10.3390/en16207165},
   issn = {19961073},
   issue = {20},
   journal = {Energies},
   keywords = {BMS,EV,LiB modeling,SOC estimation,battery pack},
   month = {10},
   publisher = {Multidisciplinary Digital Publishing Institute (MDPI)},
   title = {Modeling of the Battery Pack and Battery Management System towards an Integrated Electric Vehicle Application},
   volume = {16},
   year = {2023}
}
@article{Takyi-Aninakwa2023,
   abstract = {Due to the high nonlinearities and unstable working conditions, accurately estimating the state of charge (SOC) by the battery management system (BMS) is a major challenge in ensuring the safety and reliability of lithium-ion batteries in electric vehicles. This paper presents a deep learning network, a nonlinear autoregressive model with exogenous inputs (NARX) network with a closed-loop architecture and transfer learning mechanism, which is optimized using a proposed adaptive weighted square-root cubature Kalman filter (AWSCKF) with a moving sliding window and an adaptive weighing coefficient for SOC estimation of lithium-ion batteries. The proposed AWSCKF method is established through square-root and cubature updates to optimize the statistical value of the state estimate, error covariance, and measurement noise covariance matrices, with the ability to incorporate high nonlinearities to filter out the noise, stabilize, and optimize the final SOC. To evaluate the effectiveness of the optimized NARX network and verify the proposed AWSCKF method, battery tests are carried out using a lithium cobalt oxide battery at various charge-discharge rates and a lithium nickel cobalt manganese oxide battery at temperatures of 0 and 45 °C under five complex working conditions. The SOC accuracy of lithium-ion batteries is enhanced by the hybrid method estimation process, which is based on sensitivity analysis and adaptation to various working conditions. The comprehensive results show that the proposed NARX-AWSCKF model achieves the overall best mean absolute error, root mean square error, and mean absolute percentage error values of 0.07293%, 0.0912%, and 0.40356%, respectively, under various complex conditions. By effectively utilizing battery domain knowledge for real-world BMS applications, the proposed model outperforms other existing methods in terms of high effectiveness, robustness, and potential to boost the NARX performance.},
   author = {Paul Takyi-Aninakwa and Shunli Wang and Hongying Zhang and Yang Xiao and Carlos Fernandez},
   doi = {10.1016/j.est.2023.107728},
   issn = {2352152X},
   journal = {Journal of Energy Storage},
   keywords = {Adaptive weighted square-root cubature Kalman filter,Closed-loop architecture,Nonlinear autoregressive model with exogenous inputs,State of charge,lithium-ion battery},
   month = {9},
   publisher = {Elsevier Ltd},
   title = {A NARX network optimized with an adaptive weighted square-root cubature Kalman filter for the dynamic state of charge estimation of lithium-ion batteries},
   volume = {68},
   year = {2023}
}
@book{,
   abstract = {"Welcome to Copenhagen and the 2nd IEEE Conference on Control Technology and Applications (CCTA 2018)."--PDF welcome page. },
   isbn = {9781538676981},
   publisher = {IEEE},
   title = {2018 IEEE Conference on Control Technology and Applications (CCTA) : 21-24 August 2018},
   year = {2018}
}
@article{Yang2023,
   abstract = {To monitor and predict battery states, a battery model with accurate model parameters is important to battery management systems (BMS). However, for multi-timescale dynamic characteristics, the precision and adaptability of parameter identification of the Li-ion battery model is unsatisfactory up to now. In this paper, an improved parameter identification algorithm is proposed combining fixed memory recursive least squares (FMRLS) and fading extended Kalman filter (FEKF) which are used to obtain the fast dynamic (FD) and slow dynamic (SD) parameters of equivalent circuit model (ECM) respectively. Open-circuit voltage (OCV) is identified as a component of the SD part because of its slow dynamic nature in this algorithm. Federal urban driving schedule (FUDS) and dynamic stress test (DST) tests with different initial state of charge (SOC) and temperatures were employed for verifications, and the results show that the algorithm can track the battery terminal voltage in time and the root mean square error (RMSE) is as low as 1 mV. Meanwhile, the results reveal that the advanced SOC-OCV tests can be avoided indeed, and model parameters identified by this algorithm have good robustness in different temperatures and high consistency in different operating conditions which are significantly better than conventional algorithms.},
   author = {Zhao Yang and Xuemei Wang},
   doi = {10.1016/j.est.2022.106462},
   issn = {2352152X},
   journal = {Journal of Energy Storage},
   keywords = {Equivalent circuit model,FEKF,FMRLS,Multi-timescale characteristics,Parameter identification},
   month = {3},
   publisher = {Elsevier Ltd},
   title = {An improved parameter identification method considering multi-timescale characteristics of lithium-ion batteries},
   volume = {59},
   year = {2023}
}
@article{Khalfi2021,
   abstract = {The on-board energy storage system plays a key role in electric vehicles since it directly affects their performance and autonomy. The lithium-ion battery offers satisfactory characteristics that make electric vehicles competitive with conventional ones. This article focuses on modeling and estimating the parameters of the lithium-ion battery cell when used in different electric vehicle drive cycles and styles. The model consists of an equivalent electrical circuit based on a second-order Thevenin model. To identify the parameters of the model, two algorithms were tested: Trust-Region-Reflective and Levenberg-Marquardt. To account for the dynamic behavior of the battery cell in an electric vehicle, this identification is based on measurement data that represents the actual use of the battery in different conditions and driving styles. Finally, the model is validated by comparing simulation results to measurements using the mean square error (MSE) as model performance criteria for the driving cycles (UDDS, LA-92, US06, neural network (NN), and HWFET). The results demonstrate interesting performance mostly for the driving cycles (UDDS and LA-92). This confirms that the model developed is the best solution to be integrated in a battery management system of an electric vehicle.},
   author = {Jaouad Khalfi and Najib Boumaaz and Abdallah Soulmani and El Mehdi Laadissi},
   doi = {10.11591/ijece.v11i4.pp2798-2810},
   issn = {20888708},
   issue = {4},
   journal = {International Journal of Electrical and Computer Engineering},
   keywords = {Electric vehicles Lithium batteries Modeling Parameter estimation Vehicle driving},
   month = {8},
   pages = {2798-2810},
   publisher = {Institute of Advanced Engineering and Science},
   title = {An electric circuit model for a lithium-ion battery cell based on automotive drive cycles measurements},
   volume = {11},
   year = {2021}
}
@article{Xie2023,
   abstract = {Due to excellent power and energy density, low self-discharge and long life, lithium-ion battery plays an important role in many fields. Directed against the complexity of above noises and the strong sensitivity of the common Kalman filter algorithm to noises, the state of charge estimation of lithium-ion battery based on extended Kalman filter algorithm is investigated in this paper. Based on the second-order resistor-capacitance equivalent circuit model, the battery model parameters are identified using the MATLAB/Simulink software. A battery parameter test platform is built to test the charge-discharge efficiency, open-circuit voltage and state of charge relationship curve, internal resistance and capacitance of the individual battery are tested. The simulation and experimental results of terminal voltage for lithium-ion battery is compared to verify the effectiveness of this method. In addition, the general applicability of state of charge estimation algorithm for the battery pack is explored. The ampere-hour integral method combined with the battery modeling is used to estimate the state of charge of lithium-ion battery. The comparison of extended Kalman filter algorithm between experimental results and simulation estimated results is obtained to verify the accuracy. The extended Kalman filter algorithm proposed in this study not only establishes the theoretical basis for the condition monitoring but also provides the safe guarantee for the engineering application of lithium-ion battery.},
   author = {Jiamiao Xie and Xingyu Wei and Xiqiao Bo and Peng Zhang and Pengyun Chen and Wenqian Hao and Meini Yuan},
   doi = {10.3389/fenrg.2023.1180881},
   issn = {2296598X},
   journal = {Frontiers in Energy Research},
   keywords = {MATLAB/simulink,extended Kalman filter algorithm,lithium-ion battery,second-order resistor-capacitance (RC) equivalent circuit model,state of charge (SOC)},
   publisher = {Frontiers Media S.A.},
   title = {State of charge estimation of lithium-ion battery based on extended Kalman filter algorithm},
   volume = {11},
   year = {2023}
}
@article{Tran2021,
   abstract = {The equivalent circuit model (ECM) is a battery model often used in the battery management system (BMS) to monitor and control lithium-ion batteries (LIBs). The accuracy and complexity of the ECM, hence, are very important. State of charge (SOC) and temperature are known to affect the parameters of the ECM and have been integrated into the model effectively. However, the effect of the state of health (SOH) on these parameters has not been widely investigated. Without a good understanding of the effect of SOH on ECM parameters, parameter identification would have to be done manually through calibration, which is inefficient. In this work, experiments were performed to investigate the effect of SOH on Thevenin ECM parameters, in addition to the effect of SOC and temperature. The results indicated that with decreasing SOH, the ohmic resistance and the polarization resistance increase while the polarization capacitance decreases. An empirical model was also proposed to represent the effect of SOH, SOC, and temperature on the ECM parameters. The model was then validated experimentally, yielding good results, and found to improve the accuracy of the Thevenin model significantly. With low complexity and high accuracy, this model can be easily integrated into real-world BMS applications.},
   author = {Manh Kien Tran and Manoj Mathew and Stefan Janhunen and Satyam Panchal and Kaamran Raahemifar and Roydon Fraser and Michael Fowler},
   doi = {10.1016/j.est.2021.103252},
   issn = {2352152X},
   journal = {Journal of Energy Storage},
   keywords = {Battery management system,Battery modeling,Equivalent circuit model,Lithium-ion batteries,State of charge,State of health},
   month = {11},
   publisher = {Elsevier Ltd},
   title = {A comprehensive equivalent circuit model for lithium-ion batteries, incorporating the effects of state of health, state of charge, and temperature on model parameters},
   volume = {43},
   year = {2021}
}
@article{Tekin2024,
   abstract = {Lithium-ion batteries need to be controlled by a Battery Management System (BMS) to operate safely and efficiently. BMS controls parameters, such as current, voltage, temperature, state of charge (SoC),state of health (SoH), state of power (SoP) and etc. The battery models and several prediction algorithms that the BMS uses to carry out these checks are essential to the system's performance. Therefore, the battery model is crucial to the BMS. This model is used to optimize the performance, capacity, lifetime and safety of the battery. Using the accurate battery model for BMS and electric vehicles can improve energy efficiency, extend battery life and reduce safety risks. Therefore, it is important that the model can accurately reflect the battery behavior under different load conditions. In this study, the performance of Rint, Partnership for a New Generation of Vehicles (PNGV), Thevenin, and Dual Polarization (DP) battery models, which are widely known in the literature, to simulate static and dynamic voltage behavior is compared. A 18650 NMC battery was used for this purpose, and Hybrid Pulse Power Characterization (HPPC), Dynamic Stress Test (DST), Worldwide Harmonised Light Vehicle Test Procedure (WLTP), and Constant Current (CC) discharge tests were performed. The performance of the models for the four tests is compared. The maximum error values for WLTP are 2.98 % in Rint, 1.32 % in PNGV, 2.80 % in Thevenin, and 1.09 % in DP. Comparing the performances of models for all tests, it is found that the DP model is the most accurate model under both constant and dynamic current conditions.},
   author = {Merve Tekin and M. İhsan Karamangil},
   doi = {10.1016/j.est.2024.111327},
   issn = {2352152X},
   journal = {Journal of Energy Storage},
   keywords = {Battery management system,Battery modeling,Dual polarization model,PNGV model,Rint model,Thevenin model},
   month = {5},
   publisher = {Elsevier Ltd},
   title = {Comparative analysis of equivalent circuit battery models for electric vehicle battery management systems},
   volume = {86},
   year = {2024}
}
@article{Damodaran2024,
   abstract = {Equivalent circuit modelling (ECM) is a powerful tool to study the dynamic and non-linear characteristics of Li-ion cells and is widely used for the development of the battery management system (BMS) of electric vehicles. The dynamic parameters described by the ECM are used by the BMS to estimate the battery state of charge (SOC), which is crucial for efficient charging/discharging, range calculations, and the overall safe operation of electric vehicles. Typically, the ECM approach represents the dynamic characteristics of the battery in a mathematical form with a limited number of unknown parameters. Then, the parameters are calculated from voltage and current information of the lithium-ion cell obtained from controlled experiments. In the current work, a faster and simplified first-order resistance–capacitance (RC) equivalent circuit model was developed for a commercial cylindrical cell (LGM50 21700). An analytical solution was developed for the equivalent circuit model incorporating SOC and temperature-dependent RC parameters. The solution to the RC circuit model was derived using multiple expressions for different components like open circuit voltage (OCV), instantaneous resistance (R0), and diffusional parameters (R1 and C1) as a function of the SOC and operating temperature. The derived parameters were validated against the virtual HPPC test results of a validated physics-based electrochemical model for the voltage behavior. Using the developed RC circuit model, a polynomial expression is derived to estimate the temperature increase of the cell including both irreversible and reversible heat generation components. The temperature predicted by the proposed RC circuit model at different battery operating temperatures is in good agreement with the values obtained from the validated physics model. The developed method can find applications in (i) onboard energy management by the BMS and (ii) quicker evaluation of cell performance early in the product development cycle.},
   author = {Vijayakanthan Damodaran and Thiyagarajan Paramadayalan and Diwakar Natarajan and Ramesh Kumar C and P. Rajesh Kanna and Dawid Taler and Tomasz Sobota and Jan Taler and Magdalena Szymkiewicz and Mohammed Jalal Ahamed},
   doi = {10.3390/batteries10060215},
   issn = {23130105},
   issue = {6},
   journal = {Batteries},
   keywords = {HPPC cycle,battery state of charge,electrochemical model,equivalent circuit modelling,resistance–capacitance (RC) circuit model},
   month = {6},
   publisher = {Multidisciplinary Digital Publishing Institute (MDPI)},
   title = {Development of a Fast Running Equivalent Circuit Model with Thermal Predictions for Battery Management Applications},
   volume = {10},
   year = {2024}
}
@article{Tao2023,
   abstract = {The state-of-charge (SoC) stands as a pivotal measure for ascertaining a battery's remaining capacity. Accurate SoC estimations can meaningfully enhance a battery's operational longevity, fortify safety standards, and enrich user experience. This paper presents an improved open circuit voltage (OCV)-based partnership for a new generation of vehicle (PNGV) model, specifically tailored for estimating the SoC of LiFePO4 blade batteries. These batteries are distinctively characterized by their advantages in safety, energy density, and thermal management. The proposed model uniquely integrates the SoC-dependent property of the battery's internal resistance, facilitating a marked improvement in estimation accuracy over existing PNGV models. Experimental results underscore the capability and effectiveness of the proposed model in estimating real-time discharging curves, achieving a remarkably low relative error rate of 0.85%.},
   author = {Zhen Tao and Zhenyu Zhao and Fei Fan and Huamin Jie and Yongqi Chang and Kye Yak See},
   doi = {10.1109/IECON51785.2023.10312229},
   isbn = {9798350331820},
   issn = {25771647},
   journal = {IECON Proceedings (Industrial Electronics Conference)},
   keywords = {Blade battery,open circuit voltage (OCV),partnership for a new generation of vehicle (PNGV) model,state-of-charge (SoC) estimation},
   publisher = {IEEE Computer Society},
   title = {High Precision SoC Estimation of LiFePO4 Blade Batteries Using Improved OCV-Based PNGV Model},
   year = {2023}
}
@article{Xinyu2025,
   abstract = {With the transformation of the global energy landscape, lithium-ion batteries have become an important component in the field of new energy storage. Accurate assessment of battery status plays a crucial role in efficiently utilizing electrical energy and extending the battery's service life. The key parameters of battery status include charging state (SOC) and power state (SOP). This paper constructs an improved 2RC-PNGV battery equivalent circuit model and introduces an innovative method to enhance the dynamics of particle swarm optimization. At the same time, an adaptive H infinity (<i>∞</i>) filtering algorithm based on Sage-Husa and a temperature-constrained SOP estimation method for lithium-ion batteries is designed. Among them, the real-time dynamic particle swarm optimization algorithm adjusts the forgetting factor in each iteration; the adaptive H<i>∞</i> filtering algorithm based on Sage-Husa improves the accuracy of SOC estimation by adapting the noise covariance matrix. Moreover, the multi-parameter constrained state estimation method for lithium-ion batteries can effectively track the changes in state quantities with different durations and instantaneous values. The improved forgetting factor least squares method has an error of fewer than 0.02 volts in the voltage simulation test, with high accuracy. The adaptive H<i>∞</i> filtering algorithm based on Sage-Husa achieves higher estimation accuracy in three complex operating scenarios, ensuring that the state quantity estimation error remains below 2%. The maximum estimation error of the multi-parameter constrained state quantity estimation method is less than 84.00 watts. These research results provide a solid theoretical foundation for ensuring the safety and efficient operation of batteries.},
   author = {Yan Xinyu and Wang Shunli and Xu Tao and Cheng Liangwei and Carlos Fernandez and Frede Blaabjerg},
   doi = {10.11648/J.AJEE.20251303.14},
   issn = {2329-163X},
   issue = {3},
   journal = {American Journal of Energy Engineering 2025, Volume 13, Page 133},
   keywords = {Dynamic Particle Swarm Optimization algorithm,Estimation Strategy of SOC,H Infinity Filtering,Lithium,Power State Estimation Strategy,State Joint Estimation,ion Battery},
   month = {8},
   pages = {133-141},
   publisher = {Science Publishing Group},
   title = {Improved 2RC-PNGV Modeling and Adaptive Sage-Husa H Infinity Filtering for Battery Power State Estimation Based on Multi-Parameter Constraints},
   volume = {13},
   url = {https://www.sciencepg.com/article/10.11648/j.ajee.20251303.14},
   year = {2025}
}
